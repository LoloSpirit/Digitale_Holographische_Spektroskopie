\documentclass[11pt]{article}

% Use Times-like font for pdfLaTeX
\usepackage{mathptmx}

% Set up page geometry
\usepackage[a4paper, margin=1in]{geometry}
\usepackage[utf8]{inputenc}
\usepackage[T1]{fontenc}
\usepackage[german]{babel} % Use 'english' for English text
\usepackage{amsmath}
\usepackage{graphicx}
\usepackage{geometry}
\usepackage{booktabs} % For professional tables
\usepackage{hyperref}
\usepackage{parskip} % Space between paragraphs
\usepackage[backend=biber]{biblatex}
\addbibresource{references.bib} % BibTeX file for references

% Header and footer
\usepackage{fancyhdr}
\pagestyle{fancy}
\fancyhead[L]{Spektroskopie}
\fancyhead[C]{Digitale Holographische Spektroskopie}
\fancyhead[R]{Lorenz Saalmann}
\fancyfoot[C]{\thepage}

\begin{document}

\setlength{\parindent}{0pt}

% Document starts here

\begin{titlepage}
    \centering
    \begin{figure}
        \centering
        \includegraphics[width=0.3\textwidth]{images/jlu_logo.jpeg}
    \end{figure}
    \vspace*{2cm}
    \text{Ausarbeitung} \\
    \Large{Digitale Holographische Spektroskopie} \\
    \vspace{2cm}
    \normalsize{Lorenz Saalmann (8104072)} \\
    \vfill
    \normalsize{{Spektroskopie, SS 25}} \\
    \small{PD Dr. Arash Rahimi-Iman, Dipl.-Ing.} \\
\end{titlepage}

\newpage

\section{Einführung}
\vspace{-0.3cm}
\hspace{.0cm}\fontsize{9}{0}{\bf{AI-assisted by ChatGPT \cite{AI}}}
\vspace{0.4cm}
\\
\small
Spektroskopie befasst sich mit der Wechselwirkung von Licht und Materie. Im Fokus steht dabei die Abhängigkeit der Lichtintensität von Wellenlänge oder Frequenz. Damit lassen sich Phänomene wie Absorption, Emission und Streuung untersuchen. Da die beschriebene Wechselwirkung so grundlegend ist, findet die Spektroskopie in sehr vielen Bereichen Anwendung, wie zum Beispiel in der Chemie, Physik, Astronomie und Biologie. Sie dient der Materialanalyse, Zusammensetzungsbestimmung und der Untersuchung von physikalischen Eigenschaften von Stoffen. Konventionelle spektroskopische Verfahren basieren meist auf Intensitätsmessungen und liefern dadurch indirekte Informationen über Struktur oder Zusammensetzung eines Objekts. Insbesondere die Phaseninformation -- und damit quantitative Aussagen zur optischen Weglänge oder Topographie -- bleiben dabei unzugänglich.

Im Gegensatz dazu ermöglicht Holographie eine vollständige Erfassung des komplexen Lichtfeldes (Amplitude und Phase), wodurch eine hochauflösende dreidimensionale Rekonstruktion von Objekten möglich wird. Digitale Holographie überträgt dieses Prinzip in den digitalen Bereich: Mit modernen Bildsensoren und numerischen Rekonstruktionsverfahren lässt sich das holographisch aufgezeichnete Interferenzmuster analysieren. Kombiniert mit spektralen Messmethoden entsteht so die Digitale Holographische Spektroskopie (DHS). So können phasen-sensitive, räumliche und spektrale Informationen gleichzeitig erfasst werden, was eine tiefe Analyse des Lichtfeldes ermöglicht. Besonders in der Nanophotonik, der biomedizinischen Bildgebung und der Untersuchung optischer Materialien bietet DHS entscheidende Vorteile.
\section{Grundlagen der Holographie}

\section{Digitale Holographie}

\section{Anwendung in der Spektroskopie}

\section{Ausblick}

\printbibliography
\end{document}